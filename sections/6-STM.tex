\section{STM - Software Transactional Memory}
\begin{frame}{ZSTM}
  \begin{block}{Definizione}
    STM è uno strumento che permette di comporre singole operazioni in un'unica transazione eseguibile in maniera atomica. In \texttt{ZIO} viene rappresentato dalla struttura \texttt{ZSTM[R, E, A]}.
  \end{block}

  \begin{exampleblock}{Vantaggi}
    \begin{itemize}
      \item più potenti di \texttt{Ref}, poiché componibili
      \item esegue transazioni condizionali evitando l'utilizzo di \textit{locks}
      \item le transazioni sono libere da \textit{deadlocks} o \textit{race conditions}
    \end{itemize}
  \end{exampleblock}

  \begin{alertblock}{Limitazioni}
    \begin{itemize}
      \item non è possibile eseguire istruzioni concorrenti all'interno di una ZSTM
      \item nel caso di forte contesa vi è un calo di \textit{performance}
    \end{itemize}
  \end{alertblock}
\end{frame}

\begin{frame}{ZSTM II}
  \begin{block}{Esempio: trasferimento bancario}
    \lstinputlisting[language=Scala]{code/6a-zstm.scala}
  \end{block}
  \begin{block}{Operatori}
    \begin{itemize}
      \item \texttt{retry}: permette di rieseguire un'intera transazione
      \item \texttt{commit}: converte una transazione in uno \texttt{ZIO} \textit{effect}
    \end{itemize}
  \end{block}
\end{frame}