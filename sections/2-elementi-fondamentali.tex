\section{Elementi fondamentali}

\begin{frame}{Functional Effects}
  \begin{block}{\textit{Purely functional programming}}
    \texttt{ZIO} permette di sviluppare codice puramente funzionale, cioè i programmi possono essere descritto come composizione di espressioni che rispettano le seguenti proprietà: 
    \begin{itemize}
      \item \textbf{\textit{referential transparency}}: le espressioni sono sostituibili dai valori;
      \item \textbf{\textit{local reasoning}}: le funzioni sono autoesplicative.
    \end{itemize}
    Queste facilitano il \textit{refactoring} e la comprensione del codice.
  \end{block}

  \begin{alertblock}{Perchè i \textit{side effect} sono un problema?}
    Le proprietà appena presentate vengono meno nel caso di \textit{side effects}. Serve quindi un ponte tra questi e la programmazione puramente funzionale.
  \end{alertblock}
\end{frame}

\begin{frame}{Functional Effects II}
  \begin{block}{ZIO Core}
    \vspace{4mm}
    \begin{columns}
      \begin{column}{.5\textwidth}
        Il tipo di dato principale di \texttt{ZIO} è \texttt{ZIO[-R, +E, +A]} che consente di convertire istruzioni in \textit{functional effect} al fine di ritardarne la valutazione.
      \end{column}
      \begin{column}{.4\textwidth}
        \begin{itemize}
          \item \texttt{R} - \textit{environment type}.
          \item \texttt{E} - \textit{error type}.
          \item \texttt{A} - \textit{success type}.
        \end{itemize}
      \end{column}
    \end{columns}
    \vspace{2mm}
  \end{block}

  \begin{exampleblock}{Vantaggi}
    \begin{itemize}
      \item Separazione tra il "cosa" il programma dovrà fare dal "come" lo farà.
      \item Sviluppo incrementale di programmi complessi tramite \textbf{componibilità}.
      \item Fruizione dei vantaggi del \textit{type system} di Scala. 
      %\lstinline[language=Scala]{ZIO[Clock, Nothing, Unit]}
    \end{itemize}
  \end{exampleblock}
\end{frame}

\begin{frame}[fragile]{ZIO Constructors \& Operators}
  \begin{block}{\texttt{ZIO} constructors}
    I costruttori di \texttt{ZIO} permetto di ricondurre codice affetto da \textit{side-effects} a funzioni pure componibili. Tra i principali:
    \begin{itemize}
      \item \texttt{ZIO.succeed}: converte valori puri;
      \item \texttt{ZIO.attempt}: converte codice procedurale affetto da \textit{side-effects};
      \item \texttt{ZIO.async}: converte codice asincrono basato su \textit{callbacks};
      \item \texttt{ZIO.fromFuture}: converte una funzione che crea una \texttt{Future}.
    \end{itemize}
  \end{block}
  \begin{block}{\texttt{ZIO} operators}
    Una peculiarità di \texttt{ZIO} sono gli operatori, come \texttt{flatMap}, \texttt{foreach}, \texttt{zip}, \texttt{collectAll}, che permettono di trasformare e combinare \textit{effect} tra loro.
    \begin{itemize}
      \item \small\texttt{flatMap} + \textit{for comprehension} 
      $\rightarrow$ espressività programmazione imperativa.
    \end{itemize}
  \end{block}
\end{frame}

\begin{frame}[fragile]{ZIO Constructors \& Operators II}
  \begin{block}{Esempio: assegnazione di un \textit{task}}
    \lstinputlisting[language=Scala]{code/1a-effects.scala}
  \end{block}
  \pause
  \begin{block}{Esempio: assegnazione di un \textit{task} in \texttt{ZIO}}
    \lstinputlisting[language=Scala]{code/1b-effects.scala}
  \end{block}
\end{frame}

