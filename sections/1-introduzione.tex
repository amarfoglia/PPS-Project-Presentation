\section{Introduzione}
\begin{frame}{Cos'è ZIO?}
  \begin{columns}
    \begin{column}{.5\textwidth}
      \begin{figure}
        \centering
        \includegraphics[width=0.9\textwidth]{img/zio-logo.jpg}
        \label{ZIO logo.}
      \end{figure}
    \end{column}
    \begin{column}{.5\textwidth}
      \begin{center} 
        \texttt{ZIO} è una libreria puramente funzionale, \texttt{type-safe}, e componibile a supporto della programmazione concorrente e asincrona in Scala.
      \end{center}
    \end{column}
  \end{columns}
\end{frame}
%
\begin{frame}
    \frametitle{Perchè ZIO?}
    \begin{columns}
        \begin{column}{.5\textwidth}
            \centering{\small
              {\textbf{Concorrenza}}\break
              applicazioni concorrenti esenti da \textit{deadlocks} e \textit{race conditions}.
            }
        \end{column}
        \begin{column}{.5\textwidth}
            \centering{\small
              {\textbf{Efficienza}}\break
              cancellazione automatica di computazioni non più necessarie.
            }
        \end{column}
    \end{columns}
    
    \bigskip

    \begin{columns}
      \begin{column}{.5\textwidth}
          \centering{\small
            {\textbf{Resource-safety}}\break
            gestione automatica del ciclo di vita delle risorse.
          }
      \end{column}
      \begin{column}{.5\textwidth}
        \centering{\small
          {\textbf{Resilienza}}\break
          gestione statica degli errori, \texttt{Error channel} e meccanismi di \textit{scheduling}.
        }
      \end{column}
    \end{columns}

    \bigskip

    \begin{columns}
      \begin{column}{.5\textwidth}
          \centering{\small
            {\textbf{Testkit}}\break
            sviluppo rapido di test deterministici e \textit{type-safe}.
          }
      \end{column}
      \begin{column}{.5\textwidth}
        \centering{\small
          {\textbf{Streaming}}\break
          presenza di \textit{stream} efficienti, \textit{lazy}, \textit{resource-safe}, concorrenti e infiniti.
        }
      \end{column}
    \end{columns}
        
\end{frame}